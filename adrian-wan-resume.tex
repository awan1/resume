%!TEX TS-program = xelatex
\documentclass[]{friggeri-cv}
% \addbibresource{bibliography.bib}

\DeclareSymbolFont{extraup}{U}{zavm}{m}{n}
\DeclareMathSymbol{\varheart}{\mathalpha}{extraup}{86}

\usepackage{fontspec}
\usepackage{fontawesome}
\usepackage{enumitem}


\begin{document}
\header{adrian }{wan}
       {algorithms design \& software engineer @ Nest}


% In the aside, each new line forces a line break
\begin{aside}
  \section{contact}
    \href{mailto:adrianwan2@gmail.com?subject=Regarding%20your%20resume}{\faEnvelope}~adrianwan2@gmail.com
    % \faPhone~(610)-505-5087
    ~
    \href{http://awan1.github.io}{\faGlobe}~awan1.github.io
    \href{http://linkedin.com/in/adrianwan2}{\faLinkedin}~lnked.in/awan
    \href{http://github.com/awan1}{\faGithub}~github.com/awan1
%  \section{languages}
%    trilingual English, Cantonese, Mandarin
%    fluent Japanese
%    beginner French, ASL
  \section{software}
    {\color{red} $\varheart$} Python
    (\texttt{pandas, SciPy})
    Scala $\cdot$ MATLAB
    gRPC $\cdot$ Protobuf
    GCP $\cdot$ Kubernetes $\cdot$ Docker
    {\color{red} $\varheart$} Git
    Atlassian Stack
    (Stash, JIRA)
  \section{influences}
    Pragmatic Programmer
    Wait But Why
    Less Wrong
\end{aside}

\section{about}
I build clean, extensible, flexible codebases to make work better.

%I develop \textbf{extensible and flexible analysis tools} to produce \\
%actionable, robust, \textbf{data-driven insights}.

\section{experience}

\begin{entrylist}
  \entry
    {2015--}
    {\href{http://nest.com}{Nest}}
    {\textbf{Algorithms Design \& Software Engineer}}
    {
      \bodyspace
      \begin{my-itemize}
      	% Slideware, testing arch, real services 
        \item Created \textbf{Python prototype} of service, device, \& app interactions to \textbf{validate end-to-end behavior}:
        \begin{my-itemize}
          \item \textbf{Designed \& implemented a modular framework} supporting CLI interactions, batch testing, \& \textbf{arbitrary substitution} with real components;
          \item Enabled \textbf{novel} integration and end-to-end tests of \textbf{user-facing behavior};
          \item With understanding gained, \textbf{self-taught Scala} to assist services team, \textbf{implementing \& shipping} changes to consumer-facing services.
        \end{my-itemize}
        % Hiddenite microservice
        \item Deployed \& owned a \textbf{gRPC microservice} to buffer teams from instability:
      
        \begin{my-itemize}
          % Apps, QA, services, FT
          \item Supported $\sim$20 people across 4 teams, \textbf{balancing} contrasting needs;
          % \item Enabled services team to transition gradually from HTTP to protobuf; 
          \item Leveraged GCP \& Kubernetes for \textbf{auto-scaling} \& \textbf{no-downtime} rollouts;
          \item \textbf{Developed processes} around frequent breaking changes to ease development; used \textbf{tiered deployment} \& \textbf{extensive smoke tests} to enable \textbf{isolated testing} of affected components. Sped up debugging $\sim$5x.
        \end{my-itemize}
      
      	% Tempcomp
        \item \textbf{Re-engineered} a MATLAB script-based, labor-intensive process for temperature sensor modeling into a streamlined, \textbf{extensible Python library}:

        \begin{my-itemize}
        \item Abstracted \textbf{mathematical models} for rapid prototyping \& exploration;
        \item \textbf{Continues to be used} for customer issues \& future products.
        %\item Used across existing and future hardware to unlock product features;
        %\item \textbf{Planned and executed} as a one-month project;
        %\item Continues to be used to address customer-facing issues.
        \end{my-itemize}
      \end{my-itemize}
    }
  \entry
    {2014}
    {\href{http://nest.com}{Nest}}
    {Algorithm Design \& Data Science Intern}
    {
      \bodyspace
      \begin{my-itemize}
      \item Spearheaded Python prototyping of \textbf{data-driven customer product}:

      \begin{my-itemize}
      \item Developed, implemented, \& evaluated on-device sensor data models;
      %\item Developed, implemented, and evaluated models of on-device sensor data using \texttt{Pandas};
      \item Employed \textbf{test-driven development} to publish a modular, extensible modeling package, used within team to prototype related features;
      \item \textbf{Balanced} development with research-style data analysis.
      \end{my-itemize}

      % \item Used Agile software development principles to meet ambitious schedules, coordinating with UI/UX, cloud services, apps, and product marketing teams.
      \end{my-itemize}
    }
  \entry
    {2013--2014} %2013 Aug-Dec
    {\href{http://www.swarthmore.edu/ssx-lab}{Swarthmore Spheromak Experiment (SSX)}}
    {Research Assistant}
    {
      \bodyspace
      \begin{my-itemize}
      % \item Received the \textbf{Vandervelde-Cheung Scholarship} for summer research.
      %; was invited to return during subsequent academic semester to continue research for academic credit.

      \item Developed Python (\texttt{SciPy}, \texttt{pandas}) analyses of plasma wind-tunnel data.

      \item Received the \textbf{Outstanding Undergraduate Poster Award} at the APS Division of Plasma Physics 2013 Meeting; \textbf{coauthored} papers published in \textit{Physical Review Letters} \& in \textit{Plasma Physics and Controlled Fusion}.
      \end{my-itemize}
    }
%\employer{Swarthmore Coll. Athletics}{Fall 2012 -- Ongoing}{Assistant Manager --- Women's Volleyball}
%\begin{achievements}
%%\item Promoted to manager from previous position as statistician ($\sim$1.5 hours/week)
%\item Promotion from statistician, hired for remaining duration of attendance at Swarthmore Coll.
%\item Committed 15+ hrs/week; employed rapid, organized multi-tasking to assist practices, track player statistics, and analyze opponent offensives to inform defensive strategy.
%\item Maintained a positive team-oriented attitude, counseled teammates and mediated discussions on rocky inter-player and player-coach relationships.
%\end{achievements}
%
%\employer{Swarthmore Coll. Athletics}{Fall 2011}{Volleyball Statistician}
%\begin{achievements}
%\item Coordinated with small team under pressure to generate live comprehensive match statistics and box scores. Required clear, collected, rapid communication and cooperation.
%\item Was recognized by Volleyball Head Coach, invited to join core team as Asst. Manager.
%\end{achievements}
\end{entrylist}

\section{education}

\begin{entrylist}
	\entry
	{2011--2015}
	{B.A. {\normalfont Physics \& Computer Science}}
	{Swarthmore College, Swarthmore PA}
	{Cumu. GPA: 3.9 $\cdot$ \emph{Phi Beta Kappa} % Total: 3.887
		\smallskip
		
%		\textbf{Physics:}
%		Statistical Physics $\cdot$
%		Quantum Theory $\cdot$
%		Analytical Dynamics $\cdot$
%		\\
%		Electrodynamics $\cdot$
%		Thermodynamics \& Stat.\ Mechanics %$\cdot$
%		%Optics $\cdot$
%		%Quantum Mechanics
%		\\
%		\textbf{CS:}
%		Intro.\ Programming Languages $\cdot$
%		Algorithms $\cdot$
%		Cloud Systems \& Data \\ Networks $\cdot$
%		Databases $\cdot$
%		Operating Systems $\cdot$
%		A.I.\ $\cdot$
%		Bioinformatics %$\cdot$
%		%Intro. Computer Systems $\cdot$
%		%Data Structures \& Algorithms
	}
\end{entrylist}

\section{for more}
Publications \& more experiences are on my personal website \& on LinkedIn.

% \section{publications}

% Put your publications here!

% \printbibsection{article}{article in peer-reviewed journal}
% \begin{refsection}
%  \nocite{*}
%  \printbibliography[sorting=chronological, type=inproceedings, title={international peer-reviewed conferences/proceedings}, notkeyword={france}, heading=subbibliography]
% \end{refsection}
% \begin{refsection}
%  \nocite{*}
%  \printbibliography[sorting=chronological, type=inproceedings, title={local peer-reviewed conferences/proceedings}, keyword={france}, heading=subbibliography]
% \end{refsection}
% \printbibsection{misc}{other publications}
% \printbibsection{report}{research reports}
\end{document}
